
%**************************************************************
% Acronimi
%**************************************************************
\renewcommand{\acronymname}{Acronimi e abbreviazioni}

\newacronym[description={\glslink{apig}{Application Program Interface}}]
    {api}{API}{Application Program Interface}

\newacronym[description={\glslink{umlg}{Unified Modeling Language}}]
    {uml}{UML}{Unified Modeling Language}

\newacronym[description={\glslink{dig}{Dependency Injection}}]
    {di}{DI}{Dependency Injection}

\newacronym[description={\glslink{ideg}{Integrated Development Environment}}]
    {ide}{IDE}{Integrated Development Environment}

\newacronym[description={\glslink{sdkg}{Software Development Kit}}]
    {sdk}{SDK}{Software Development Kit}

\newacronym[description={\glslink{jitg}{Just In Time}}]
    {jit}{JIT}{Just In Time}

\newacronym[description={\glslink{aotg}{Ahead Of Time}}]
    {aot}{AOT}{Ahead Of Time}

\newacronym[description={\glslink{uuidg}{Universally Unique Identifier}}]
    {uuid}{UUID}{Universally Unique Identifier}

\newacronym[description={\glslink{daog}{Data Access Object}}]
    {dao}{DAO}{Data Access Object}

\newacronym[description={\glslink{dtog}{Data Transfer Object}}]
    {dto}{DTO}{Data Transfer Object}

\newacronym[description={\glslink{jsong}{JavaScript Object Notation}}]
    {json}{JSON}{JavaScript Object Notation}

\newacronym[description={\glslink{pojog}{Plain Old Java Object}}]
    {pojo}{POJO}{Plain Old Java Object}

%**************************************************************
% Glossario
%**************************************************************
\renewcommand{\glossaryname}{Glossario}

\newglossaryentry{apig}
{
    name=\glslink{api}{API},
    text=Application Programming Interface,
    sort=api,
    description={In informatica con il termine \emph{Application Programming Interface (API)} (dall'inglese, interfaccia di programmazione di un'applicazione) si indica ogni insieme di procedure disponibili al programmatore, di solito raggruppate a formare un set di strumenti specifici per l'espletamento di un determinato compito all'interno di un certo programma. La finalità è ottenere un'astrazione, di solito tra l'hardware e il programmatore o tra software a basso e quello ad alto livello semplificando così il lavoro di programmazione}
}

\newglossaryentry{umlg}
{
    name=\glslink{uml}{UML},
    text=UML,
    sort=uml,
    description={Nell'ingegneria del software \emph{UML, Unified Modeling Language} (dall'inglese, linguaggio di modellazione unificato) è un linguaggio di modellazione e specifica basato sul paradigma object-oriented. L'\emph{UML} svolge un'importantissima funzione di ``lingua franca'' nella comunità della progettazione e programmazione a oggetti. Gran parte della letteratura di settore usa tale linguaggio per descrivere soluzioni analitiche e progettuali in modo sintetico e comprensibile a un vasto pubblico}
}

\newglossaryentry{dig}
{
    name=\glslink{di}{Dependency Injection},
    text=dependency injection,
    sort=dependencyinjection,
    description={Nell'ingegneria del software per \emph{Dependency Injection} (dall'inglese, iniezione delle dipendenze) è una tecnica che permette a un oggetto di ricevere le proprie dipendenze dall'esterno senza doversene occupare nella sua costruzione. Questo garantisce una migliore separazione dei compiti le classi e una maggiore facilità in fase di test}
}

\newglossaryentry{restg}
{
    name=\glslink{restg}{REST API},
    text=REST API,
    sort=rest,
    description={Le REST API (meglio definite RESTful API) sono API che seguono i principi definiti per l'architettura Representational State Transfer (REST), tipicamente implementate usando HTTP come protocollo di comunicazione}
}

\newglossaryentry{codebaseg}
{
    name=\glslink{codebaseg}{Codebase},
    text=codebase,
    sort=codebase,
    description={Si intende l'intera collezione di codice sorgente necessaria allo sviluppo di un prodotto software}
}

\newglossaryentry{tsg}
{
    name=\glslink{tsg}{TypeScript},
    text=TypeScript,
    sort=typescript,
    description={È un linguaggio di scripting superset di JavaScript che lo sostituisce ma lo integra aggiungendo i tipi statici e permettendo quindi di evitare più errori a runtime grazie alla possibilità di fare una più efficiente analisi statica. Tutto il codice JavaScript è di fatto codice TypeScript valido. Il codice TypeScript può essere a sua volta compilato per produrre del codice JavaScript}
}

\newglossaryentry{ideg}
{
    name=\glslink{ide}{Integrated Development Environment},
    text=IDE,
    sort=ide,
    description={Per \emph{IDE, Integrated Development Environment} (dall'inglese, ambiente di sviluppo integrato) si intende un software utilizzato sia per scrivere il codice sorgente di un prodotto software che per svolgere altre attività che altrimenti richiederebbero l'utilizzo di comandi da terminale, come la compilazione, l'analisi statica, l'esecuzione di test ma anche per svolgere attività come il debugging e il controllo di strumenti correlati come emulatori}
}

\newglossaryentry{frameworkg}
{
    name=\glslink{frameworkg}{Framework},
    text=framework,
    sort=framework,
    description={Un \emph{framework} è un insieme di strumenti software che forniscono allo sviluppatore utilizzatore una maggiore facilità di accesso a certe funzionalità di un linguaggio di programmazione o di un ambiente, come per esempio un sistema operativo}
}

\newglossaryentry{sdkg}
{
    name=\glslink{sdkg}{Software Development Kit},
    text=Software Development Kit,
    sort=sdk,
    description={Per \emph{SDK, Software Development Kit} (dall'inglese, kit per lo sviluppo software) si intende un insieme di strumenti per lo sviluppo e il rilascio di software che lo utilizza. Un SDK contiene ciò che normalmente contiene anche un framework, ma include anche la documentazione e gli strumenti per poter compilare e/o eseguire il codice delle applicazioni che vengono scritte con le librerie presenti nel SDK}
}

\newglossaryentry{assetg}
{
    name=\glslink{assetg}{Asset},
    text=asset,
    sort=asset,
    description={Un \emph{asset} in \emph{Flutter} è un file multimediale, un file di testo oppure un font che viene dichiarato nella configurazione di un progetto e caricato all'avvio dell'applicazione. Gli \emph{asset} sono disponibili in sola lettura}
}

\newglossaryentry{jitg}
{
    name=\glslink{jitg}{Just In Time},
    text=Just In Time,
    sort=jit,
    description={Per \emph{JIT, Just In Time} (dall'inglese, appena in tempo) si intende un tipo di compilatore che compila il codice sorgente a runtime (ovvero lo interpreta)}
}

\newglossaryentry{aotg}
{
    name=\glslink{aotg}{Ahead Of Time},
    text=Ahead Of Time,
    sort=aot,
    description={Per \emph{AOT, Ahead Of Time} (dall'inglese, prima del tempo) si intende un tipo di compilatore che compila il codice sorgente convertendolo in codice macchina in modo che il sistema ospite possa eseguirlo nativamente (ossia senza virtualizzazione)}
}

\newglossaryentry{bestpracticeg}
{
    name=\glslink{bestpracticeg}{Best practice},
    text=best practice,
    sort=bestpractice,
    description={Una \emph{best practice} (dall'inglese, buona pratica) è un modo di agire che ha dimostrato nel tempo di essere il migliore in un certo ambito}
}

\newglossaryentry{uuidg}
{
    name=\glslink{uuidg}{Universally Unique Identifier},
    text=UUID,
    sort=uuid,
    description={Un \emph{UUID, Universally Unique Identifier} (dall'inglese, identificativo univoco universale) rappresenta un tipo di identificativo per identificare univocamente un'informazione in assenza di un sistema di coordinamento. Ciò significa che entità completamente distinte e di sistemi diversi sono comunque identificabili se hanno un UUID, in quanto - in pratica - univoco}
}

\newglossaryentry{dtog}
{
    name=\glslink{dtog}{Data Transfer Object},
    text=DTO,
    sort=datatransferobject,
    description={Un \emph{DTO, Data Transfer Object} (dall'inglese, oggetto per il trasferimento dati) rappresenta un tipo di classe molto semplice contenente dati ottenuti da remoto o da altre API strutturati per un loro agevole passaggio fra varie entità all'interno dell'architettura software}
}

\newglossaryentry{daog}
{
    name=\glslink{daog}{Data Access Object},
    text=DAO,
    sort=dataaccessobject,
    description={Un \emph{DAO, Data Access Object} (dall'inglese, oggetto per l'accesso ai dati) rappresenta un tipo di classe in cui vengono elaborati dati ottenuti da remoto, da altre API, ricevuti come argomento sotto forma di DTO e ritorna altri dati, spesso sotto forma di DTO}
}

\newglossaryentry{jsong}
{
    name=\glslink{jsong}{JavaScript Object Notation},
    text=JSON,
    sort=javascriptobjectnotation,
    description={Per \emph{JSON} si intende un tipo di notazione tipicamente usata per rappresentare in JavaScript gli oggetti, usata comunemente dalle API di tipo RESTful come formato di interscambio dati, al pari di altri come ad esempio XML}
}

\newglossaryentry{designpatterng}
{
    name=\glslink{designpatterng}{Design pattern},
    text=design pattern,
    sort=designpattern,
    description={Un \emph{design pattern} è una soluzione software generica e quindi riutilizzabile che risolve in un modo efficiente ed efficace un problema che si presenta spesso durante la progettazione di un sistema software. I design pattern principali sono quelli creazionali (che risolvono problemi comuni nella creazione di oggetti), strutturali (che implementano soluzioni a problemi nella gestione di oggetti), comportamentali (che offrono soluzioni all'utilizzo di oggetti)}
}

\newglossaryentry{pojog}
{
    name=\glslink{pojog}{POJO},
    text=POJO,
    sort=pojo,
    description={Un \emph{Plain Old Java Object, POJO} è un tipo di classe che viene definita in Java in cui non ci sono dipendenze in ingresso da altre librerie, possibilmente non hanno annotazioni e sono serializzabili, e le proprietà sono accessibili attraverso \emph{getter} e \emph{setter}}
}


% acronimo più glossario
\newacronym[description={\glslink{testoConG}{TestoXEsteso}}]
    {TESTO_DA_CATTURARE}{ACRONIMO}{TestoXEsteso}

\newglossaryentry{testoConG}
{
    name=\glslink{TESTO_DA_CATTURARE}{ACRONIMO},
    text=TestoXEsteso,
    sort=TESTO_DA_ORDINARE,
    description={Descrizione}
}

% solo glossario
\newglossaryentry{testoConGg}
{
    name=\glslink{testoConG}{TestoXEsteso},
    text= TestoXEsteso,
    sort=TESTO_DA_ORDINARE,
    description={Descrizione}
}