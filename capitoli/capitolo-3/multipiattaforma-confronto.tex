%**************************************************************
\section{Framework multipiattaforma a confronto}
\label{sec:multipiattaforma-confronto}

Una volta scelto di voler realizzare un'applicazione con una tecnologia multipiattaforma (che da adesso in poi chiameremo \gls{frameworkg})  è stato necessario anche capire qual era quella più adatta agli scopi. Per questa scelta sono state tenute in considerazione:
\begin{itemize}
    \item linguaggio di programmazione utilizzato;
    \item team che realizza il \gls{frameworkg} (se è realizzato da un'organizzazione solida e conosciuta è più probabile che sia stabile e ben funzionante, considerando che delle risorse stanno venendo impiegate per il suo sviluppo);
    \item età, ovvero da quanto tempo è disponibile, poiché meno è recente più è possibile che vi sia maturità e stabilità;
    \item performance garantite, perché sebbene in situazioni di normale utilizzo ci si aspetta che un'applicazione funzioni comunque correttamente e senza problemi, è necessario assicurarsi che anche sotto stress la reazione fornita sia buona e che non avvengano crash o malfunzionamenti per incapacità del \gls{frameworkg} in uso;
    \item disponibilità di librerie per aggiungere funzionalità al \gls{frameworkg} (il che dipende molto dall'interesse che suscita negli sviluppatori di queste e la facilità nel loro sviluppo);
    \item facilità di accedere alle funzionalità native del dispositivo (fotocamera, GPS, ma anche \gls{api} di sistema). 
\end{itemize}
Considerate queste caratteristiche, vengono di seguito elencate le tecnologie che sono state vagliate per l'utilizzo nel progetto di stage, affrontando tutti gli aspetti appena indicati.

%**************************************************************
\subsection{React Native}
\label{subsec:react-native}

\emph{React Native} è un \gls{frameworkg} rilasciato inizialmente da Facebook nel 2015 ed è mantenuto attivamente dalla comunità\footnote{Per comunità si intende la comunità open source, ovvero tutti coloro che contribuiscono ad un progetto open source.}, essendo un progetto open source.\\
Il nome deriva dalla principale libreria impiegata nel \gls{frameworkg}, che è \emph{React}: essendo questa libreria sviluppata per essere utilizzata in progetti scritti in JavaScript, anche \emph{React Native} di conseguenza usa JavaScript.\\
L'interfaccia grafica viene formata attraverso componenti, composte da un misto di JavaScript e un linguaggio di markup simile a HTML. Queste componenti sono però molto basiche, e quindi componenti più complesse vanno realizzate dallo sviluppatore secondo le proprie necessità.\\
Essendo molto conosciuto come \gls{frameworkg}, trovare una libreria di terze parti che implementa una certa funzionalità desiderata, non presente all'interno dello strumento, non è un problema. Piuttosto, si rischia di dipendere troppo da queste in quanto, come detto in precedenza, le componenti base sono veramente essenziali.\\
Nuovamente, l'accesso alle funzionalità native del dispositivo sono implementate principalmente attraverso librerie di terze parti e solo in parte da \gls{api} integrate nel \gls{frameworkg}, potendo far nascere alcuni problemi in caso queste non siano più portate avanti dai loro principali sviluppatori.\\
Nonostante tutto, però, le performance garantite sono buone, in quanto dal codice sorgente viene prodotta una vera app con codice nativo.

%**************************************************************
\subsection{Xamarin}
\label{subsec:xamarin}

\emph{Xamarin} è un \gls{frameworkg} rilasciato inizialmente da Xamarin nel 2011, ora di proprietà di Microsoft, e mantenuto dallo stesso team e dalla comunità.\\
È basato sul \gls{frameworkg} \emph{.NET} di Microsoft e il linguaggio che viene usato è C\#. La parte di \emph{business logic} è completamente gestita con codice C\#.\\
L'implementazione dell'interfaccia utente dipende dalla versione dello strumento:
\begin{itemize}
    \item \emph{Xamarin} permette di creare interfacce grafiche native, creandole separatamente per ciascuna piattaforma di destinazione, raddoppiando di fatto il lavoro per lo sviluppatore in questa parte dell'app;
    \item \emph{Xamarin.Forms} invece permette di scrivere dei file XAML unici, separati dalla parte di \emph{business logic}.
\end{itemize}
L'accesso alle funzionalità native è supportato pienamente e le performance garantite sono ancora una volta molto buone.\\
Un'imposizione di \emph{Xamarin} è l'utilizzo dell'\gls{ide} Visual Studio, poiché tutte le fasi di sviluppo, testing, debugging e compilazione passano attraverso questo strumento.\\
Sebbene il \gls{frameworkg} sia gratuito, Visual Studio non lo è per usi commerciali: le versioni \emph{Professional} ed \emph{Enterprise} sono a pagamento.

%**************************************************************
\subsection{Flutter}
\label{subsec:flutter}

\emph{Flutter} è un \gls{frameworkg} e un \gls{sdk} rilasciato da Google in versione stabile nel 2018 ed è mantenuto attivamente sia dal team di Google che si occupa del \gls{frameworkg} che dalla comunità.\\
L'interfaccia grafica è formata quasi interamente da widget, di cui ne è disponibile una gamma molto vasta. Fra questi widget sono inoltre disponibili due principali macro-gruppi:
\begin{itemize}
    \item i widget di tipo \emph{Material}, pensati per Android perché seguono le linee guida del \emph{Material Design}\footcite{site:material-design} di Google;
    \item i widget di tipo \emph{Cupertino}, perché emulano le componenti dell'interfaccia dei sistemi Apple (il nome \emph{Cupertino} prende spunto dalla città dove ha sede Apple).
\end{itemize}
È relativamente nuovo ma offre già una buona stabilità. Il linguaggio utilizzato è Dart, creato anch'esso da Google, ma che a differenza di Flutter è più maturo (rilasciato per la prima volta nel 2011) ed è mantenuto attivamente sia da Google che dalla comunità.\\
\emph{Flutter} non è conosciuto come \emph{React Native}, ma l'interesse verso di esso è crescente, per cui sempre più sviluppatori si interessano al progetto.\\
Tante funzionalità sono già incluse all'interno del \gls{frameworkg} stesso, altre sono disponibili come librerie esterne mantenute dallo stesso team o da quello di Dart, altre sono invece di terze parti. Le librerie vengono pubblicate in una piattaforma centralizzata chiamata \emph{Pub.dev}\footcite{site:pub-dev}, agevolando notevolmente la ricerca.\\
L'accesso alle \gls{api} di sistema non è quindi un problema e in ogni caso è possibile colmare le mancanze utilizzando dei meccanismi per accedere al codice nativo, senza particolari problemi, comunicando attraverso i cosiddetti \emph{Platform Channel}.\\
Una menzione degna di nota su \emph{Flutter}, nonostante non sia ancora del tutto stabile e in sviluppo attivo, è la possibilità di usare lo stesso identico \gls{frameworkg} e quindi lo stesso progetto sia per sistemi desktop come Windows, MacOS e sistemi Linux, che per siti web.\\
Come per \emph{React Native} e \emph{Xamarin}, viene compilato codice nativo quindi le applicazioni realizzate, se ben progettate e codificate, possono essere molto fluide e reattive.

%**************************************************************
\subsection{Ionic}
\label{subsec:ionic}

\emph{Ionic} è un \gls{frameworkg} rilasciato da Ionic in versione stabile nel 2013 ed è mantenuto attivamente principalmente dal team di Ionic assieme alla comunità.\\
A differenza dei precedenti tre, \emph{Ionic} non crea un'app nativa bensì crea una web app che viene poi eseguita in un contenitore, una cosiddetta \emph{webview}, che altro non è che un browser semplificato, permettendo infine di creare un'applicazione per dispositivi mobili.\\
Per svolgere questo compito si serve di \emph{Apache Cordova}\footcite{site:apache-cordova} o \emph{Capacitor}\footcite{site:capacitor}.
Offre allo sviluppatore numerose componenti già pronte da utilizzare che, usate in combinazione con numerosi \gls{frameworkg} per la creazione di applicazioni web come \emph{Angular}\footcite{site:angular} ad esempio, permettono di creare applicazioni anche molto complesse.\\
Producendo sostanzialmente una web-app i linguaggi a disposizione per lo sviluppatore sono HTML, CSS e JavaScript.\\
Come ulteriore conseguenza di ciò, tutte le librerie disponibili per JavaScript e per le tre tecnologie sopracitate sono utilizzabili anche in questo contesto.\\
L'accesso alle \gls{api} di sistema è permesso grazie ad \emph{Apache Cordova} e \emph{Capacitor}, servendosi di librerie che fanno da involucro alle funzionalità native. In caso ci siano funzionalità mancanti, è possibile crearle e aggiungerle al proprio \gls{codebaseg}.\\
In quanto web app, la sua velocità è all'incirca quella che ci si può aspettare da un sito web, di conseguenza se un obiettivo del prodotto è avere buone performance anche in situazioni di stress, \emph{Ionic} non è la scelta migliore.

%**************************************************************
\subsection{Web app}
\label{subsec:web-app}

Un'altra possibile vagliata è stata quella di realizzare una web app. Quanto detto per \emph{Ionic} segue anche per la web app, in quanto è quello che viene prodotto anche per esso.\\
La differenza principale starebbe nell'accesso alle funzionalità native del dispositivo, che dovrebbero passare attraverso la gestione del browser in uso e inoltre, per motivi di sicurezza, non tutte potrebbero essere disponibili.
Infine, l'applicazione andrebbe comunque rilasciata attraverso un dominio disponibile in rete oppure inserita in un contenitore come la precedentemente citata \emph{webview}.
