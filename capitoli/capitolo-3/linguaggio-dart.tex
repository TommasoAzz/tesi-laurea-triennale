%**************************************************************
\section{Dart, il linguaggio usato in Flutter}
\label{sec:linguaggio-dart}

Dart è il linguaggio che viene utilizzato da \emph{Flutter} nelle librerie del \gls{frameworkg}, nella parte di visualizzazione dei widget a schermo e anche nella parte di business logic.

%**************************************************************
\subsection{Caratteristiche del linguaggio}
\label{subsec:caratteristiche-linguaggio}

Dart è un linguaggio orientato agli oggetti e fortemente tipato. Nonostante sia fortemente tipato è possibile fare l'inferenza di tipo, quindi:
\begin{lstlisting}
var numero = 5;
int numero = 5;
\end{lstlisting}
sono istruzioni equivalenti ed entrambe hanno tipo statico e dinamico \texttt{int}.\\
Ogni variabile è un riferimento ad un oggetto, anche i numeri, le funzioni e \texttt{null} (simile a quanto avviene nei linguaggi che usano la Java Virtual Machine).
Tutti i tipi ereditano dal tipo base \texttt{Object}.\\
Esiste un tipo speciale, \texttt{dynamic}, che segnala al compilatore che non si è a conoscenza del tipo dinamico della variabile (il tipo statico è invece \texttt{dynamic}) e che può essere un tipo qualsiasi.\\
Sono disponibili le procedure e le routine (le cosiddette funzioni, slegate dal concetto di classe), come in C/C++ e JavaScript, a differenza di Java. Il punto di ingresso di un programma scritto in Dart (e di un'applicazione con \emph{Flutter}) è la funzione \texttt{void main()}.

%**************************************************************
\subsubsection{Classi e oggetti}
\label{subsubsec:classi-oggetti}

Oltre a quanto già affermato, in Dart esistono solo due modificatori di accesso: \emph{public} e \emph{private}.\\
Un campo dati o un metodo di una classe sono pubblici di default (non esiste una keyword \emph{public} o una con lo stesso significato).\\
Per indicare che un campo dati o un metodo è privato il suo identificatore deve cominciare con \textbf{\_} (trattino basso) (anche in questo caso, non esiste una keyword \emph{private} o una con lo stesso significato).\\
Non esisto il concetto di \emph{protected}.\\
Gli stessi concetti di modificatori di accesso si applicano anche ai nomi delle classi e delle funzioni in un file, quindi se una classe o una funzione è pubblica lo sarà ovunque; se è privata lo sarà ovunque tranne che nel file in cui è specificata.\\

%**************************************************************
\subsubsection{Subtyping}
\label{subsubsec:subtyping}

Una classe in Dart può essere \textbf{concreta} oppure \textbf{astratta} (ovvero non istanziabile perché definita a livello di classe o perché dispone di metodi non implementati).\\
Una classe viene dichiarata esattamente come in molti altri linguaggi:
\begin{lstlisting}
class NomeClasse {
    // Definizione campi dati.
    ...
    // Definizione metodi.
    ...
}
\end{lstlisting}
Se la classe è astratta è sufficiente indicare \texttt{abstract class} al posto di \texttt{class}.\\
Un metodo senza implementazione (astratto) non ha alcuna keyword che lo identifica: semplicemente, invece di specificare il corpo del metodo fra parentesi graffe (o con la \emph{arrow annotation}, \texttt{=>}, per corpi di funzione con una singola istruzione) si termina la definizione del metodo con un punto e virgola.\\
Il concetto di \textbf{interfaccia} è presente, ma non come in linguaggi come Java: in Dart \textbf{ogni classe definisce implicitamente un'interfaccia}, per cui una classe può essere indifferentemente ereditata e/o implementata.

%**************************************************************
\subsubsection{Mixin}
\label{subsubsec:mixin}

I \textbf{mixin} sono delle classi che contengono metodi usabili da altre classi senza dover essere una loro superclasse. Sono a tutti gli effetti delle classi concrete dal punto di vista dell'implementazione (tutti i metodi sono implementati e c'è la possibilità di avere dei campi dati) ma non possono essere istanziati direttamente. Possono invece essere integrati nella definizione di altre classi tramite la keyword \texttt{with}.\\
Un esempio di definizione ed uso di un mixin è il seguente:
\begin{lstlisting}
mixin NomeMixin {
  // Definizione campi dati.
  ...

  // Definizione metodi.
  void metodoDelMixin() {
    ...
  }
  ...
}

class NomeClasseConMixin with NomeMixin {
    // Definizione campi dati.
    ...

    // Definizione metodi.
    void metodo() {
        metodoDelMixin();
        ...
    }
    ...
}
\end{lstlisting}

%**************************************************************
\subsubsection{Supporto a programmazione asincrona}
\label{subsubsec:supporto-programmazione-asincrona}

Gran parte delle operazioni che richiedono di operare con I/O (file, rete, ecc.), eseguire calcoli complessi o eseguire operazioni che non possono/riescono ritornare immediatamente, sono asincrone.\\
L'asincronia in Dart è rappresentata da due tipi principali:
\begin{itemize}
    \item \texttt{Stream}, che rappresenta un "flusso" a cui ci si deve mettere in ascolto per ottenerne i dati (è molto simile agli stream delle \emph{Stream API} in Java) che potrebbero presentarsi in successione;
    \item \texttt{Future}, che rappresenta dati che "verranno ritornati in futuro" (dove per "in futuro" si intende che il tempo impiegato può essere più o meno lungo e non è possibile assumerne la durata) una volta sola.
\end{itemize}
Di seguito è mostrata una correlazione fra Future e Stream, asincroni, e la loro corrispondente versione sincrona:
{
\renewcommand{\arraystretch}{1.5}
\begin{longtable}{|c|c|c|}
    \hline
    -                       & \textbf{Sincrono}      & \textbf{Asincrono}   \\\hline
    \endhead
    \textbf{Valore singolo} & \texttt{int}           & \texttt{Future<int>} \\\hline
    \textbf{Collezione}     & \texttt{Iterable<int>} & \texttt{Stream<int>} \\\hline
    \caption{Correlazione fra campi dati sincroni e asincroni}
\end{longtable}
}
Una funzione (o metodo) che ritorna \texttt{Future} o \texttt{Stream} non viene subito eseguita, ne viene solamente registrata la richiesta di esecuzione e espletata non appena possibile.
Affinché si sappia quali operazioni eseguire una volta che l'operazione asincrona è stata eseguita va registrato un \emph{callback} tramite il metodo \texttt{then} (per \texttt{Future}) e \texttt{listen} (per \texttt{Stream}).\\
Per evitare di registrare \emph{callback} sono presenti due keyword, \texttt{async} (da indicare nella dichiarazione di un metodo) e \texttt{await} (da indicare prima dell'espressione che invoca un metodo asincrono). Queste due keyword permettono di mantenere l'aspetto asincrono ma al contempo di rendere molto più leggibile il codice sorgente.\\
Di seguito viene mostrato un esempio di utilizzo di \texttt{Future} registrando un \emph{callback}:
\begin{lstlisting}
void fun() {
  // Creazione di un riferimento a una computazione
  // che termina in futuro.
  Future<int> myFutureWithCallback = futureOperation();

  // Registrazione di un callback che viene invocato
  // al completamento del futuro.
  myFuture.then((intValue) {
    // Eseguito dopo, quando futureOperation
    // ha finito la sua computazione.
    print("Stampato dopo. " + intValue);
  });

  print("Stampato prima.");
}
\end{lstlisting}
Il successivo esempio mostra invece l'utilizzo di \texttt{Future} con \texttt{async}/\texttt{await}. Per poterli paragonare, viene riadattato l'esempio precedente:
\begin{lstlisting}
void fun() async {
  // Creazione di un riferimento a una computazione
  // che termina in futuro.
  Future<int> myFutureWithCallback = await futureOperation();
    
  print("Stampato prima.");
  print("Stampato dopo " + intValue);

}
\end{lstlisting}
Entrambi gli esempi stamperanno "Stampato prima. Stampato dopo." seguiti dal numero intero ritornato da \texttt{futureOperation}.

%**************************************************************
\subsection{Compilazione ed esecuzione}
\label{subsec:compilazione-esecuzione}

Una volta aver discusso su alcune caratteristiche tecniche del linguaggio, è bene approfondire come Dart può essere compilato ed eseguito.\\
È già stato affermato che Dart può essere sia interpretato che compilato, tutto questo attraverso lo stesso \gls{sdk}.
Le possibilità offerte sono quattro:
\begin{itemize}
    \item Dart Native, che può interpretare e/o compilare codice Dart per dispositivi mobili (quello che è stato fatto per lo stage) ma anche desktop;
    \item Dart Web, che può interpretare e/o compilare codice Dart per web (ovvero generare JavaScript).
\end{itemize}

%**************************************************************
\subsubsection{Dart Native}
\label{subsubsec:dart-native}

Questa parte dell'\gls{sdk} si occupa della generazione di codice eseguibile per dispositivi mobili e/o desktop (per questo motivo si parla di "native").\\
Ad occuparsi di questo compito è la \emph{Dart VM} che, un po' come la JVM per Java, esegue al suo interno il codice Dart.\\
La \emph{Dart VM} è in grado di interpretare Dart tramite un compilatore \gls{jit} e viene usato in fase di debugging perché decisamente più veloce di una normale compilazione.
Inoltre, permette di supportare l'\emph{Hot Reload} e l'\emph{Hot Restart} di \emph{Flutter}.\\
In fase di rilascio dell'applicazione il codice non viene interpretato ma compilato dal compilatore \emph{Dart AOT} che compila, come dice il nome, \gls{aot}.
Il codice compilato viene poi eseguito all'interno di un runtime che include, tra l'altro, il garbage collector.

%**************************************************************
\subsubsection{Dart Web}
\label{subsubsec:dart-web}
Questa parte dell'\gls{sdk} si occupa della generazione di codice eseguibile per browser (per questo motivo si parla di "web").\\
Ci sono ancora una volta due strumenti, uno per l'interpretazione del codice sorgente durante il debugging e un altro per generare codice JavaScript.\\
Il primo strumento si chiama \emph{dartdevc}, ed è paragonabile alla \emph{Dart VM}. Il secondo si chiama \emph{dart2js} ed è invece l'equivalente del compilatore \emph{Dart AOT}.