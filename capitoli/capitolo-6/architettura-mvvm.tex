%**************************************************************
\section{Model View ViewModel, l'architettura scelta}
\label{sec:architettura-mvvm}

%**************************************************************
\subsection{Ragioni della scelta}
\label{subsec:ragioni-scelta}

Dopo aver analizzato le varie caratteristiche delle architetture presentate nella precedente sezione, è stata scelta l'architettura \emph{Model View ViewModel}.\\
Innanzitutto, \emph{MVVM} è consigliato per alcune delle altre tecnologie multipiattaforma che sono state ipotizzate per realizzare il prodotto di stage, per via della natura dichiarativa con cui viene implementata l'interfaccia utente.
Inoltre, l'utilizzo del \gls{designpatterng} Observer doppiamente permette di garantire un notevole disaccoppiamento fra le tre parti dell'architettura, permettendo di conseguenza una maggior facilità nel testing.\\
Per quanto riguarda le altre architetture, \emph{Model View Presenter} è stata scartata principalmente per la necessità del \emph{Presenter} di aggiornare la vista, che non si trova molto in sintonia con il metodo di rappresentazione dell'interfaccia di \emph{Flutter}.\\
\emph{Model View Controller} non è stato scelto perché il \emph{Model} avrebbe avuto troppi oneri: nel sistema da realizzare per lo stage era necessario memorizzare i dati ottenuti dalle \gls{restg} per poterli utilizzare per più scopi in più viste, quindi richiedendo che in qualche punto dell'architettura ci sia questo tipo di disponibilità (possibilmente non nella \emph{View}).\\
Infine, \emph{Clean Architecture}, come affermato nella sotto-sezione in cui la si illustrava, è certamente una buona architettura ma difficilmente implementabile con l'esperienza avuta fino ad ora. È comunque un'architettura valida e con i dovuti aggiustamenti \emph{MVVM} potrebbe essere adattata a questa.

%**************************************************************
\subsection{Organizzazione dei package}
\label{subsec:organizzazione-package}

%**************************************************************
\subsubsection{Package it.tecsen.smacs}
\label{subsubsec:it-tecsen-smacs}

\begin{figure}[!h] 
  \centering 
  \includegraphics[width=1.0\columnwidth]{capitolo-6/organizzazione-package/it-tecsen-smacs} 
  \caption{Diagramma del package \texttt{it.tecsen.smacs}}
\end{figure}
Il seguente package racchiude tutto il codice sorgente dell'applicazione (non è stato richiesto l'uso del \emph{Platform Channel}), librerie escluse.\\
È suddiviso in più sotto-package seguendo il pattern architetturale Model View ViewModel, per cui sono presenti (come si può vedere nel diagramma sottostante) i package \texttt{model}, \texttt{viewmodel} e \texttt{view}.\\
\textbf{N.B.} Le dipendenze evidenziate nel diagramma sono quelle principali. Ad esempio, quasi tutti i package dipendono da \texttt{utils} e \texttt{config}, dato che contengono rispettivamente classi di utilità e di configurazione, ma non sono state segnate.\\
I package \texttt{screen} e \texttt{widget} fanno sempre parte del concetto \texttt{view} del pattern, ma sono separate dal package \emph{view} per ragioni organizzative:
\begin{itemize}
  \item \texttt{screen} contiene le classi che implementano i widget radice di ogni schermata dell'applicazione. Ognuno di questi widget è ridotto al minimo essenziale per poter poi utilizzare un widget dichiarato nel package \texttt{view} come resto del \emph{Widget tree};
  \item \texttt{widget} contiene le classi che implementano alcuni widget usati dalle classi in \texttt{screen} e \texttt{view}, a scopo di riutilizzo.
\end{itemize}
I package \texttt{model}, \texttt{viewmodel} e \texttt{view} contengono le classi per svolgere il loro ruolo all'interno dell'architettura.\\
Il package \texttt{utils} contiene classi con metodi di utilità (per esempio per il parsing di UUID, per l'hashing di stringhe, per la gestione di date, del logging e altro).\\
Il package \texttt{config} contiene classi con la configurazione dell'app, di alcuni plugin (come quello per la pubblicazione di notifiche), dell'internazionalizzazione e mappe per analizzare i dati ricevuti dalle \gls{restg}.\\
Il package \texttt{webserviceclient} contiene l'implementazione di un semplice wrapper di alcuni metodi della libreria \emph{http}\footcite{site:http-library} per facilitarne l'uso all'interno delle classi del package \texttt{model}.\\
Infine, \texttt{SMacsApp} è la classe che implementa il widget radice dell'intera applicazione e si serve delle classi \texttt{Routes} e \texttt{Providers} in cui sono presenti rispettivamente la configurazione della navigazione all'interno dell'app e le dipendenze di tutte le classi del package \texttt{viewmodel}.


%**************************************************************
\subsubsection{Package it.tecsen.smacs.screen}
\label{subsubsec:it-tecsen-smacs-screen}

% L'immagine dovrebbe essere qui ma per motivi di spazio viene spostata.
In questo package vi sono tutte le classi che implementano i widget radice di ogni schermata dell'applicazione, come è stato precedentemente illustrato in "\hyperref[subsubsec:it-tecsen-smacs]{Package it.tecsen.smacs}".\\
Ogni screen implementa uno o più casi d'uso, disponibili per la consultazione nell'appendice "\hyperref[cap:analisi-dei-requisiti]{Analisi dei requisiti}".
% \\
\clearpage
\begin{figure}[!h]
  \centering 
  \includegraphics[width=1.0\columnwidth]{capitolo-6/organizzazione-package/screen} 
  \caption{Diagramma del package \texttt{it.tecsen.smacs.screen}}
\end{figure}
In particolare:
{
\renewcommand{\arraystretch}{1.5}
\begin{longtable}{|c|c|}
    \hline
    \textbf{Classe (widget)} & \textbf{Caso (e sotto-casi) d'uso implementati} \\\hline
    \endhead
    SplashScreen & UC01\\\hline
    AuthScreen & UC02 \\\hline
    DeviceListScreen & UC03, UC15 \\\hline
    TrafficControllerScreen & UC04, UC05, UC07, UC09, UC10 \\\hline
    DeviceInfoScreen & UC06 \\\hline
    EventControllerScreen & UC08 \\\hline
    NotificationScreen & UC11 \\\hline
    SettingsScreen & UC12 \\\hline
    AppInfoScreen & UC13 \\\hline
    DebugModeScreen & UC14 \\\hline
    \caption{Correlazione fra classi del package \texttt{screen} e casi d'uso implementati}
\end{longtable}
}


%**************************************************************
\subsubsection{Package it.tecsen.smacs.view}
\label{subsubsec:it-tecsen-smacs-view}

\begin{figure}[!h]
  \centering 
  \includegraphics[width=1.0\columnwidth]{capitolo-6/organizzazione-package/view} 
  \caption{Diagramma del package \texttt{it.tecsen.smacs.view}}
\end{figure}
In questo package vi sono tutte le classi che implementano i widget che visualizzano il contenuto per soddisfare i casi d'uso raccolti.\\
Ogni classe implementa uno o più casi d'uso, disponibili per la consultazione nell'appendice "\hyperref[cap:analisi-dei-requisiti]{Analisi dei requisiti}".\\
In particolare:
{
\renewcommand{\arraystretch}{1.5}
\begin{longtable}{|c|c|}
    \hline
    \textbf{Classe (widget)} & \textbf{Caso (e sotto-casi) d'uso implementati} \\\hline
    \endhead
    SplashView & UC01\\\hline
    AuthView & UC02 \\\hline
    DeviceListView & UC03 \\\hline
    DeviceOverviewView & UC04.1, UC04.2, UC04.3, UC04.4, UC04.5, UC05 \\\hline
    DiagnosticsView & UC04.5, UC04.6, UC07.3 \\\hline
    ControlPanelView & UC04.5, UC07.1, UC07.2, UC07.4 \\\hline
    TrafficControllerView & UC04 e UC07 (sotto-casi esclusi), UC04.7, UC09, UC10 \\\hline
    DeviceInfoView & UC06 \\\hline
    EventControllerView & UC08 \\\hline
    NotificationView & UC11 \\\hline
    SettingsView & UC12 \\\hline
    AppInfoView & UC13 \\\hline
    DebugModeView & UC14 \\\hline
    DrawerView & UC15 \\\hline
    \caption{Correlazione fra classi del package \texttt{view} e casi d'uso implementati}
\end{longtable}
}

%**************************************************************
\subsubsection{Package it.tecsen.smacs.viewmodel}
\label{subsubsec:it-tecsen-smacs-viewmodel}

\begin{figure}[!h]
  \centering 
  \includegraphics[width=1.0\columnwidth]{capitolo-6/organizzazione-package/viewmodel} 
  \caption{Diagramma del package \texttt{it.tecsen.smacs.viewmodel}}
\end{figure}
In questo package vi sono tutte le classi che implementano i viewmodel (ossia i model per le view) che permettono alle classi del package \texttt{view} di mostrare agevolmente i dati a schermo.\\
Per ogni viewmodel esiste un'interfaccia\footnote{Da qui in avanti in questa sezione, quando si utilizza il termine "interfaccia", si intende una classe astratta con tutti metodi astratti. Viene usato quindi il significato tipico di Java e non quello di Dart.} (tutte le classi che terminano con VM) e un'implementazione (le altre rimanenti).\\
Le classi \texttt{HomeFacade} e \texttt{AppSettingsFacade} non sono implementazioni di interfacce. Entrambi, come suggerisce il loro nome (\emph{facade}), sono classi che implementano l'omonimo \gls{designpatterng} per raggruppare funzionalità comuni necessarie alle classi del package \texttt{view}.
Il primo implementa funzionalità comuni per \texttt{SplashView} e \texttt{AuthView}. Il secondo contiene metodi e proprietà per la gestione delle impostazioni dell'applicazione, tra cui la presenza o meno di notifiche, lo stato del task in background e la durata del suo timeout di accensione.\\
Ogni interfaccia di questo package viene usata da una sola classe del package \texttt{view}, e queste non hanno accesso all'implementazione concreta. I \emph{facade} vengono invece usati in più classi dello stesso package.


%**************************************************************
\subsubsection{Package it.tecsen.smacs.utils}
\label{subsubsec:it-tecsen-smacs-utils}

\begin{figure}[!h]
  \centering 
  \includegraphics[width=1.0\columnwidth]{capitolo-6/organizzazione-package/utils} 
  \caption{Diagramma del package \texttt{it.tecsen.smacs.utils}}
\end{figure}
In questi package ci sono strumenti di utilità creati per evitare di duplicare quanto più possibile codice.\\
Ogni classe ha il suo scopo, riassunto di seguito:
\begin{itemize}
  \item \texttt{CryptoUtils} include metodi per la cifratura di stringhe;
  \item \texttt{DateUtils} include metodi per ottenere date formattate in più modi;
  \item \texttt{FileIO} è un \emph{mixin} che viene utilizzato per memorizzare su file alcuni dati dell'applicazione dalle classi del package \texttt{it.tecsen.smacs.model.repository};
  \item \texttt{UuidUtils} include metodi per la codifica e la decodifica di \gls{uuid};
  \item \texttt{FutureUtils} include metodi per l'esecuzione sicura di funzioni che ritornano istanze di tipo \texttt{Future};
  \item \texttt{SynopticUtils} include metodi per la codifica e la decodifica di dati ottenuti dalle \gls{restg} per la visualizzazione dei dati del sinottico (casi d'uso UC04 e UC05);
  \item \texttt{UiUtils} include metodi di utilità generici per la creazione dell'interfaccia grafica.
\end{itemize}

%**************************************************************
\subsubsection{Package it.tecsen.smacs.model}
\label{subsubsec:it-tecsen-smacs-model}

\begin{figure}[!h]
  \centering 
  \includegraphics[width=1.0\columnwidth]{capitolo-6/organizzazione-package/model} 
  \caption{Diagramma del package \texttt{it.tecsen.smacs.model}}
\end{figure}
Questo package implementa il modello dell'applicazione.\\
Si può dividere in più aree con compiti specifici, rappresentate dai sotto-package:
\begin{itemize}
  \item \texttt{api} contiene classi che utilizzano il wrapper del package \texttt{webserviceclient} per interrogare le \gls{restg} e ritornare il risultato sotto forma di \gls{dto};
  \item \texttt{exception} contiene eccezioni che possono essere generate dalle classi presenti in questo package;
  \item \texttt{repository} contiene interfacce e classi per memorizzare su supporto persistente (le implementazioni usano dei file come supporto persistente) i dati ricevuti dai \gls{dao};
  \item \texttt{dao} contiene tutti le classi \gls{dao} che sono necessarie ai membri del package \texttt{viewmodel} per funzionare. Utilizza i package \texttt{api} e \texttt{repository} per ottenere i dati da ritornare sotto forma di \gls{dto};
  \item \texttt{dto} contiene tutti le classi \gls{dto} che sono necessarie al funzionamento dei package \texttt{api}, \texttt{dao} e \texttt{viewmodel}.
\end{itemize}
Sebbene \texttt{api} e \texttt{dao} ritornino entrambi \gls{dto}, c'è una differenza:
\begin{itemize}
  \item il primo non fa nessuna elaborazione sui dati, semplicemente li ottiene dalle \gls{restg} sotto forma di \gls{json}, ne fa il parsing e restituisce quanto ottenuto;
  \item il secondo elabora i \gls{dto} ottenuti dal primo e li ritorna elaborati (per fare un esempio, una lista ricevuta da \texttt{api} potrebbe essere ordinata, filtrata e poi ritornata).
\end{itemize}

%**************************************************************
\subsubsection{Package it.tecsen.smacs.webserviceclient}
\label{subsubsubsec:it-tecsen-smacs-webserviceclient}

\begin{figure}[!h]
  \centering 
  \includegraphics[width=1.0\columnwidth]{capitolo-6/organizzazione-package/webserviceclient} 
  \caption{Diagramma del package \texttt{it.tecsen.smacs.webserviceclient}}
\end{figure}
In questo package vi sono tutte le classi che implementano il wrapper della libreria \emph{http}, come è stato precedentemente illustrato in "\hyperref[subsubsec:it-tecsen-smacs]{Package it.tecsen.smacs}".


%**************************************************************
\subsubsection{Package it.tecsen.smacs.widget}
\label{subsubsubsec:it-tecsen-smacs-widget}

Il package \texttt{widget} non è stato riportato sotto forma diagrammatica per due ragioni:
\begin{itemize}
  \item perché è troppo vasto e dispersivo;
  \item perché contiene solo classi il cui scopo è permettere quanto più riuso di codice possibile per la parte di interfaccia utente.
\end{itemize}

%**************************************************************
\subsubsection{Package it.tecsen.smacs.config}
\label{subsubsubsec:it-tecsen-smacs-config}

Il package \texttt{config} non è stato riportato sotto forma diagrammatica perché come indicato in in "\hyperref[subsubsec:it-tecsen-smacs]{Package it.tecsen.smacs}" contiene principalmente configurazioni per il funzionamento di altre classi.

%**************************************************************
\subsection{Il design pattern Observer: Provider, ChangeNotifier e Consumer}
\label{subsec:observer-provider-changenotifier}

È stato detto che una delle caratteristiche dell'architettura MVVM è la presenza di un doppio Observer.\\
L'Observer è un \gls{designpatterng} formalizzato dalla "Gang of Four" nel libro \emph{Design Patterns - Elementi per il riuso di software ad oggetti}.\\
Permette:
\begin{itemize}
  \item di avere dipendenze di tipo "1 a molti" fra oggetti, permettendo di avere consistenza in maniera agevole;
  \item divide gli oggetti possessori di stato nella categoria \emph{Subject} e quelli che dipendono da questo stato in \emph{Observer};
  \item quando un \emph{Subject} aggiorna il suo stato notifica i propri \emph{Observer} (internamente dispone di un riferimento per ciascuno di questi, ma non sa nulla di loro per via dell'astrazione).
\end{itemize}
Di seguito viene presentato un esempio per illustrare come è stato implementato il \gls{designpatterng} Observer all'interno del prodotto.
L'esempio si riferisce alla parte del sistema in cui un \emph{ViewModel} ha ricevuto dai \gls{dao} da cui dipende una nuova lista e quindi notifica qualsiasi \emph{View} in ascolto.
Solo alcune parti (quelle fondamentali ai fini dell'esempio) vengono riportate.

%**************************************************************
\subsubsection{Il Subject: ChangeNotifier}
\label{subsubsec:subject-changenotifier}

In \emph{Flutter} (e non Dart, perché è propriamente una caratteristica del \gls{frameworkg}) un \emph{Subject} è realizzabile molto semplicemente aggiungendo nella firma di una classe il mixin \texttt{ChangeNotifier}, come mostrato di seguito:

\begin{lstlisting}
class DeviceList with ChangeNotifier implements DeviceListVM {
  ...

  DeviceListDAO _deviceListDAO;
  
  ...
}
\end{lstlisting}
In questa porzione di classe sono messi in vista anche \texttt{DeviceListVM}, che è la dipendenza di cui necessita la \emph{View} e \texttt{DeviceListDAO} che è invece una dipendenza di questo \emph{ViewModel}.\\
Nel metodo in cui viene aggiornato lo stato avviene anche la notifica degli \emph{Observer}:
\begin{lstlisting}
@override
Future<void> syncDeviceList({final bool forceDownload = false}) async {
    ...

    await _deviceListDAO.syncDeviceList(_token, _deviceTree, forceDownload: forceDownload);

    notifyListeners();
}
\end{lstlisting}
Come si può notare, dopo aver chiamato il metodo \texttt{syncDeviceList} di \texttt{DeviceListDAO} a riga 5, viene invocato il metodo \texttt{notifyListeners} che appartiene al \emph{mixin} \texttt{ChangeNotifier}.\\
Una volta ricevuta la notifica, uno dei primi metodi che viene richiamato dalla vista è quello che ritorna la cardinalità della lista che è appena stata scaricata (successivamente seguono altre chiamate), in base alla selezione dell'utente:
\begin{lstlisting}
@override
Future<int> get selectedDeviceListLength async {
  List<Device> devices;

  if (_activeDisplayMode == DisplayMode.ALPHABETICAL_ORDER) {
    devices = _deviceListDAO.alphabeticalOrderDeviceList;
  } else {
    devices = _deviceListDAO.warningDeviceList;
  }

  if (_nearbyDevicesMode) {
    devices = await _deviceListDAO.nearbyDeviceList(deviceList);
  }

  return devices.length;
}
\end{lstlisting}
I due costrutti \texttt{if} servono a verificare che tipo di lista va ritornata all'utente (in base alle sue precedenti selezioni), per invocare il metodo corretto.

%**************************************************************
\subsubsection{Il Subject: Provider}
\label{subsubsec:subject-provider}

Come è stato già illustrato al capitolo 5 sulla "\hyperref[cap:dependency-injection]{Dependency Injection}", un'istanza di \texttt{DeviceListVM} va messa a disposizione dei widget figli attraverso un \emph{Provider} come antenato nel \emph{Widget tree}.\\
In questo caso, visto che \texttt{DeviceListVM} usa \texttt{ChangeNotifier} ed ha delle dipendenze esterne (due, per l'esattezza), viene istanziato un \texttt{ChangeNotifierProxyProvider2}.
\begin{lstlisting}
class DeviceListScreen extends StatelessWidget {
  ...

  @override
  Widget build(BuildContext context) {
    return ChangeNotifierProxyProvider2<IdentityDAO, DeviceListDAO, DeviceListVM>(
      // Viene creata l'istanza ma non viene usata 
      // (non ha i riferimenti ai DAO).
      create: (_) => DeviceList(),
      // Viene aggiornata l'istanza con i nuovi DAO.
      update: (_, identityDao, deviceListDao, deviceListVM) {
        deviceListVM.deviceListDAO = deviceListDao;
        deviceListVM.identityDAO = identityDao;
        return deviceListVM;
      },

      ...
    );
  }
}
\end{lstlisting}
In \texttt{ChangeNotifierProxyProvider2} il parametro \texttt{update} gestisce il caso in cui le dipendenze di \texttt{DeviceListVM} chiamino anche loro \texttt{notifyListeners}.

%**************************************************************
\subsubsection{L'Observer: Consumer}
\label{subsubsec:observer-consumer}

Un altro modo per chiamare il \gls{designpatterng} Observer è "Producer-Consumer", in cui il \emph{Producer} corrisponde al \emph{Subject} e l'\emph{Observer} corrisponde al \emph{Consumer}.\\
Senza troppa fantasia, un widget che in \emph{Flutter} si occupa di agire da \emph{Consumer} ha proprio questo nome: si registra come \emph{Observer} rispetto ad un \emph{ChangeNotifier} e ogni volta che riceve una notifica ricostruisce il suo sotto albero (la porzione di \emph{Widget tree} radicata nel suo primo discendente).
% Extra a capo per arrivare a fine pagina
\clearpage

\begin{lstlisting}
class DeviceListScreen extends StatelessWidget {
  ...

  @override
  Widget build(BuildContext context) {
    ...
  
    return ChangeNotifierProxyProvider2<IdentityDAO, DeviceListDAO, DeviceListVM>(
      child: Consumer<DeviceListVM>(
        // Il terzo parametro, noRedraw, corrisponde ad un widget che non va ricostruito nel caso DeviceListVM emetta notifiche.
        builder: (_, deviceListVM, noRedraw) {
          return ScreenViewSafeContainer(
            child: DeviceListView(),
            bottomAppBar: FutureBuilder<int>(
            future: deviceListVM.selectedDeviceListLength,
            ...
          );
        }
        ...
      ),
    );
  }
}
\end{lstlisting}
Come si può notare si è continuato l'esempio di prima (è la stessa classe e lo stesso metodo \texttt{build}).\\
Come affermato poc'anzi, ad ogni \texttt{notifyListeners} invocato da \texttt{DeviceListVM}, verrà ricostruito il sottoalbero radicato nel primo discendente disponibile del widget \texttt{Consumer}, ovvero \texttt{ScreenViewSafeContainer} (propriamente viene invocato nuovamente il \emph{callback} \texttt{builder} a riga 11, ma è un dettaglio del \gls{frameworkg} che non cambia il principio di funzionamento).