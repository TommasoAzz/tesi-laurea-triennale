%**************************************************************
\section{Considerazioni finali sull'esperienza di stage}
\label{sec:considerazioni-finali}

Uno stage a conclusione di un ciclo di studi ritengo sia immensamente importante.\\
Dopo aver visto, nel corso degli studi, diversi aspetti dell'informatica, più o meno pratici, appresi più linguaggi e tecnologie e "concluso" con il progetto del corso di Ingegneria del software, permette di chiudere un cerchio.
Offre allo studente di applicare quanto imparato fino a quel momento e di concretizzarlo in un progetto elaborato assieme ad un'azienda, coniugando le esigenze educative, dello studente, e lavorative, dell'azienda.\\
In particolare, lo studente può rendersi conto di come si affrontano realmente i problemi in ambito lavorativo, quali sono le necessità e i rapporti fra persone all'interno di un'azienda.
In base a dove si svolge lo stage, può pur essere che si scoprano degli aspetti nella relazione fra l'azienda e i propri clienti e, sempre nel caso specifico dell'informatica, come le richieste di un cliente vengano prese in considerazione per poterle trasformare in un prodotto software che soddisfi le sue necessità.\\
Un altro aspetto fondamentale che si impara in parte già durante gli studi, ma in maniera ancora più concentrata durante lo stage, è la gestione dei tempi: avendo un tempo strettamente finito, bisogna saper coniugare gli obiettivi da raggiungere con le proprie disponibilità temporali.
Per fare un esempio, non è possibile sviluppare un prodotto con una tecnologia rispetto ad un'altra se farlo impiegherebbe più tempo di quanto disponibile, come non è altrettanto possibile non far caso a ritardi o a problemi che si incontrano durante il percorso.
Dal punto di vista lavorativo, si impara quindi a ragionare all'interno di certi vincoli, da un certo punto di vista necessari anche per il rispetto di sé stessi, e di sapervici rientrare ottenendo il miglior risultato possibile.