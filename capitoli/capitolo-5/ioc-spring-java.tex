%**************************************************************
\section{Inversion of Control: il framework Spring in Java}
\label{sec:ioc-spring-java}

\emph{Spring} è uno dei \gls{frameworkg} più conosciuti ed usati per sviluppare software in Java. È diviso in tanti progetti più piccoli ma che hanno tutti alla base un efficiente meccanismo di \gls{dig} attraverso l'\emph{Inversion of Control}.\\
L'\emph{Inversion of Control (talvolta abbreviato in IoC)} è un meccanismo che permette di migliorare ancora di più l'idea alla base della \gls{dig}, che consiste nel tenere il comportamento di una classe separato dalla risoluzione delle sue dipendenze, avendo un contenitore che gestisce autonomamente questo meccanismo.\\
Il compito di uno sviluppatore resta quello di dichiarare le dipendenze, dove necessarie, e registrare, attraverso dei meccanismi definiti dal \gls{frameworkg}, come permettere di ottenerle.\\
Al collegamento fra le dipendenze stesse e le classi che invece le richiedono ci pensa il meccanismo di \emph{IoC}.\\

%**************************************************************
\subsection{Creazione di una dipendenza}
\label{subsec:creazione-dipendenza-spring}

In \emph{Spring} ci sono vari modi per dichiarare al meccanismo di \emph{IoC} che un'istanza di una classe deve essere usata come dipendenza. Di seguito vengono mostrati i due principali, gli altri derivano da questi.

%**************************************************************
\subsubsection{Utilizzo di POJO: l'annotazione @Bean}
\label{subsubsec:utilizzo-pojo-annotazione-bean-spring}

Il primo modo che è disponibile per creare una dipendenza è usando l'annotazione di \emph{Spring} \texttt{@Bean}. Tipicamente questo metodo è utilizzato per ritornare oggetti singoli che vengono istanziati agevolmente all'interno di un metodo (i cosiddetti \gls{pojo}), come configurazioni di librerie o di altre parti dell'applicazione.\\
\begin{lstlisting}
class XYZ {
    ...
    @Bean
    public Dipendenza getDipendenza() {
        return new Dipendenza();
    }
}
\end{lstlisting}
Facendo riferimento all'esempio appena presentato, marchiando un'istanza o un metodo di una classe (che deve essere sotto il controllo di \emph{Spring}, nel prossimo esempio viene spiegato come) con \texttt{@Bean}, il meccanismo di \emph{IoC} si salva il riferimento all'istanza ritornata dal metodo \texttt{getDipendenza} e, quando deve istanziare un oggetto che la richiede, gliela passa come argomento.

%**************************************************************
\subsubsection{Utilizzo di classi: l'annotazione @Component}
\label{subsubsec:utilizzo-classi-annotazione-component-spring}

Un altro modo, e da questo ne derivano altri del tutto simili, è marchiare una classe con l'annotazione \texttt{@Component}.
Le differenze con gli altri sono riconducibili all'annotazione usata: \texttt{@Controller}, \texttt{@RestController}, \texttt{@Service} e \texttt{@Configuration}.
Tutte le annotazioni elencate aiutano il \gls{frameworkg} nel lavoro di risoluzione delle dipendenze, ma hanno di fondo lo stesso funzionamento.

\begin{lstlisting}
@Component
class Dipendenza {
    ...
}
\end{lstlisting}
In questo modo, quando una classe o un metodo richiedono una dipendenza di tipo \texttt{Dipendenza}, il meccanismo di \emph{IoC} crea un'istanza di questa classe e la passa come argomento.

%**************************************************************
\subsection{Richiesta di una dipendenza}
\label{subsec:richiesta-dipendenza-spring}

Il modo per informare il meccanismo di \emph{IoC} di necessitare di una dipendenza è utilizzando una delle due annotazioni \texttt{@Autowired} o \texttt{@Inject}.
\begin{lstlisting}
class Ricevente {
    @Autowired
    private Dipendenza nonDichiarata;

    private Dipendenza dichiarata;

    @Autowired
    public Ricevente(Dipendenza dipendenza) {
        dichiarata = dipendenza;
    }
}
\end{lstlisting}
In questo esempio vengono mostrati due modi per ottenere la risoluzione della dipendenza:
\begin{itemize}
    \item il primo, corrispondente alla variabile \texttt{nonDichiarata}, è ottenere la dipendenza senza dichiararla.
    È assolutamente uguale al successivo, ma non dichiarandola nel costruttore o in un setter, è più facile da dimenticare (per esempio, in fase di testing);
    \item la seconda, corrispondente alla variabile \texttt{dichiarata}, è ottenere la dipendenza dal costruttore, facendo quella che viene chiamata \emph{constructor injection}.
\end{itemize}

%**************************************************************
\subsection{Osservazioni sul procedimento}
\label{subsec:osservazioni-procedimento-spring}

Questo è quanto è necessario fare per la \gls{dig} in \emph{Spring}. Un'eccezione è fatta per il punto di ingresso, ovvero la classe con il metodo \texttt{main}, che va annotata con \texttt{@SpringBootApplication} (ma che è automaticamente incluso alla generazione di un nuovo progetto).

