%**************************************************************
\section{Requisiti e obiettivi finali}
\label{sec:requisiti-obiettivi-finali}

Gli obiettivi definiti per lo stage erano i seguenti:
\begin{itemize}
    \item sviluppare un'applicazione multipiattaforma sulla base dell'applicazione già in uso presso l'azienda, disponibile per solo sistema operativo Android;
    \item l'applicazione doveva reperire i dati dai server aziendali attraverso \gls{restg};
    \item l'applicazione doveva inoltre attivarsi in background, su richiesta dell'utente, ad intervalli di tempo predefiniti per verificare la presenza di nuove informazioni da comunicare.
\end{itemize}
Oltre a questi obiettivi elencati, obbligatori, ce n'erano due ulteriori, ma secondari:
\begin{itemize}
    \item internazionalizzare l'applicazione (in italiano e in inglese, ma facendo in modo che altre lingue possano essere aggiunte);
    \item comprendere come generare i pacchetti da distribuire ai clienti dell'azienda per l'installazione dell'applicazione.
\end{itemize}
Un obiettivo che si è ritenuto necessario introdurre durante lo stage, senza però formalizzarlo, è stato quello di effettuare un restyling grafico dell'interfaccia utente dell'applicazione, con un occhio sia all'esperienza utente che all'accessibilità. Alcune considerazioni su questo punto vengono proposte in \hyperref[sec:confronto-precedente-app-nuova]{Confronto fra la precedente app e quella nuova}.\\
Per i requisiti basati sui casi d'uso elaborati durante lo stage, consultare l'"\hyperref[cap:analisi-dei-requisiti]{Analisi dei requisiti}" in appendice.