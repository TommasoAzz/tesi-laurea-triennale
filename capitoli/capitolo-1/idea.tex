%**************************************************************
\section{L'idea}
\label{sec:idea}
In un mondo in cui l'informatica e le telecomunicazioni (in inglese, \emph{Information and Communication Technologies - ICT}) pervadono sempre più ogni ambito, automatizzando e migliorando, in efficienza ed efficacia, i sistemi in cui vengono applicate, i sistemi di trasporto non potevano non essere parte di questo cambiamento.\\
L'ingegneria dei trasporti è quella parte dell'ingegneria che si occupa di creare soluzioni per il trasporto delle merci, delle persone, tra cui la realizzazione di infrastrutture e veicoli.\\
L'innovazione tecnologica ha fatto sì che i sistemi di trasporto (in inglese, \emph{Transportation Systems}) venissero integrati con la telematica, creando dei "sistemi di trasporto intelligenti" (in inglese, \emph{Intelligent Transportation Systems}).\\
Un \emph{Intelligent Transportation System (ITS)} è quindi il risultato dell'applicazione dell'ICT alle infrastrutture per i trasporti ed ai veicoli, al fine di rendere più efficiente i sistemi di trasporto.\\
Più in dettaglio, possiamo scomporre il termine \emph{ITS} e analizzarlo parola per parola:
\begin{itemize}
    \item \emph{Intelligent}, si riferisce ad entità che hanno l'abilità di apprendere, adattarsi a nuove situazioni e usare le informazioni che raccolgono per migliorare la loro efficienza operativa;
    \item \emph{Transportation}, entità che collaborano per il movimento di merci e persone;
    \item \emph{System}, entità che svolgono attività in base a regolamenti interni e con un fine comune.
\end{itemize}
Un \emph{ITS} è importante per molteplici ragioni:
\begin{itemize}
    \item per ridurre le congestioni del traffico;
    \item per permettere di risparmiare tempo e consumare meno carburante;
    \item per ridurre le emissioni inquinanti;
    \item per permettere un abbassamento del numero di incidenti;
    \item per favorire una rete di trasporti in generale più sicura e che garantisca meno stress.
\end{itemize}
Con un \emph{ITS} è possibile:
\begin{itemize}
    \item gestire le Zone a Traffico Limitato (ZTL);
    \item gestire il traffico e il transito dei veicoli;
    \item dare priorità al transito dei mezzi pubblici;
    \item raccogliere dati sul traffico;
    \item dare informazioni in tempo reale tramite segnalatori acustici e visivi;
    \item gestire i parcheggi e la loro capienza;
    \item gestire le stazioni meteorologiche;
    \item controllare la qualità dell'aria;
    \item collezionare dati per inviarli a dei server per poterli analizzare e monitorare mediante sale di controllo.
\end{itemize}