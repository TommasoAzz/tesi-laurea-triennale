%**************************************************************
\section{Organizzazione del testo}
\label{sec:organizzazione-testo}

\begin{description}
    \item[{\hyperref[cap:introduzione]{Il primo capitolo}}] offre una panoramica sul contesto applicativo in cui si pone il progetto realizzato proposto dall'azienda ospitante.

    \item[{\hyperref[cap:descrizione-stage]{Il secondo capitolo}}] descrive come è stato organizzato lo stage, la pianificazione del lavoro e gli obiettivi da raggiungere.
    
    \item[{\hyperref[cap:scelta-tecnologie]{Il terzo capitolo}}] spiega come è stata fatta la scelta delle tecnologie usate per realizzare il prodotto di stage.
    
    \item[{\hyperref[cap:confronto-paradigmi]{Il quarto capitolo}}] tratta della differenza fra il paradigma di programmazione impiegato dalle tecnologie scelte e quello usato in altre tecnologie apprese nel corso degli studi;
    
    \item[{\hyperref[cap:dependency-injection]{Il quinto capitolo}}] illustra della differenza fra come la \gls{di} viene implementata nelle tecnologie scelte e in quelle di altre tecnologie apprese nel corso degli studi.
    
    \item[{\hyperref[cap:prodotto-stage]{Il sesto capitolo}}] dimostra come è stata scelta l'architettura impiegata per realizzare il prodotto finale e come è stato implementato.

    \item[{\hyperref[cap:conclusioni]{Il settimo e ultimo capitolo}}] propone infine alcune conclusioni sull'esperienza svolta, tra cui delle valutazioni sullo stage e sul prodotto realizzato.
\end{description}
Gli acronimi, le abbreviazioni e i termini ambigui o di uso non comune menzionati vengono definiti nel glossario, situato alla fine del presente documento.